\documentclass{article}
\usepackage{graphicx} % Required for inserting images

\title{Algorithms Project Proposal}
\author{Team 46 L4 (Ghulam Mustafa and Syeda Wania Hussain)}

\begin{document}

\maketitle



\section{Paper Details}
\begin{itemize}
\item \textbf{Title:} Efficient Data Structure and Algorithms for Minimum Transfers in Public Transportation
\item \textbf{Authors:} Mithinti Srikanth, Aditya Pandey, G. Ramakrishna
\item \textbf{Conference:} 2024 Sixteenth International Conference on Contemporary Computing (IC3-2024)
\item \textbf{Year:} 2024
\item \textbf{DOI/Link:} \url{https://dl.acm.org/doi/10.1145/3675888.3676057}
\end{itemize}

\section{Summary}
This paper addresses the \textbf{minimum transfers problem} in public transportation networks, which aims to find the least number of vehicle switches required to travel from one location to another. The authors introduce a new data structure called the \textbf{Temporal Graph} and use it to construct the \textbf{Temporal Paths Preservation (TPP) Graph}, which eliminates invalid routes and maintains only time-respecting paths. The paper develops five algorithms: \textit{Single Queue, No Queue, Multiple Queue, Priority Queue, and Multiple Priority Queue}, to efficiently compute the minimal transfer from a source node to all other destinations. The effectiveness of these algorithms is evaluated using real-world public transport datasets.

\section{Justification}
This paper is highly relevant as it tackles a critical real-world problem of optimizing public transport routes. The proposed algorithms enable efficient route planning, which in turn can \textbf{improve passenger convenience, reduce waiting times, and enhance public transport usability}.

\section{Implementation Feasibility}
The paper provides pseudocode for all the mentioned algorithms(except for Mutliple Priority Queue approach), along with clear explanations of the data structures used and the workings of each algorithm. This makes implementation feasible and structured.The GTFS(General Transit Feed Specification) datasets that the authors have used are also publicly available for testing.

\section{Team Responsibilities}
\begin{itemize}
\item \textbf{Analyzing the paper and algorithms:} Both members
\item \textbf{Implementation of algorithms:} Split between both members
\item \textbf{Data processing:} Ghulam Mustafa
\item \textbf{Analysis:} Wania
\item \textbf{Writing and documentation:} Both members
\end{itemize}

\end{document}

